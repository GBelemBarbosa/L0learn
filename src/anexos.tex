%LTeX: language=pt-BR
\appendix

\chapter{Comparação adicional de estratégias de validação cruzada}
\label{chap:anexos}

Neste apêndice, apresentamos comparações adicionais das estratégias de validação cruzada para diferentes cenários de correlação e dimensionalidade, complementando a discussão da \autoref{fig:cv_comp}.

\begin{figure}[ht]
  \centering
  \subbottom[NSPG+CPSI1: Similaridade.]{%
    \includegraphics[width=0.32\linewidth]{\cvPath SPG/const-0.5-2000/Comp_Sim-NSPG+CPSI1_Comparison_const-0.5-2000.png}}
  \subbottom[L0Learn+CPSI1: Similaridade.]{%
    \includegraphics[width=0.32\linewidth]{\cvPath L0LearnPSI1/const-0.5-2000/Comp_Sim-L0LearnCPSI1_Comparison_const-0.5-2000.png}}
  \subbottom[NSPG-PGCCD-CPSI1: Similaridade.]{%
    \includegraphics[width=0.32\linewidth]{\cvPath SPGpCDSS/const-0.5-2000/Comp_Sim-NSPG+PGCCD+CPSI1_Comparison_const-0.5-2000.png}}\\
  \subbottom[NSPG-CPSI1: Tempo.]{%
    \includegraphics[width=0.32\linewidth]{\cvPath SPG/const-0.5-2000/Comp_Time-NSPG+CPSI1_Comparison_const-0.5-2000.png}}
  \subbottom[L0Learn-CPSI1: Tempo.]{%
    \includegraphics[width=0.32\linewidth]{\cvPath L0LearnPSI1/const-0.5-2000/Comp_Time-L0LearnCPSI1_Comparison_const-0.5-2000.png}}
  \subbottom[NSPG-PGCCD-CPSI1: Tempo.]{%
    \includegraphics[width=0.32\linewidth]{\cvPath SPGpCDSS/const-0.5-2000/Comp_Time-NSPG+PGCCD+CPSI1_Comparison_const-0.5-2000.png}}
  \caption{Comparação das estratégias de validação cruzada (Correlação constante, $\rho=0.5,\ p=2000,\ k^\dagger=100,\ \text{SNR}=10$).}
  \label{fig:cv_comp_app1}
\end{figure}

\begin{figure}[ht]
  \centering
  \subbottom[NSPG-CPSI1: Similaridade.]{%
    \includegraphics[width=0.32\linewidth]{\cvPath SPG/const-0.9-1000/Comp_Sim-NSPG+CPSI1_Comparison_const-0.9-1000.png}}
  \subbottom[L0Learn-CPSI1: Similaridade.]{%
    \includegraphics[width=0.32\linewidth]{\cvPath L0LearnPSI1/const-0.9-1000/Comp_Sim-L0LearnCPSI1_Comparison_const-0.9-1000.png}}
  \subbottom[NSPG-PGCCD-CPSI1: Similaridade.]{%
    \includegraphics[width=0.32\linewidth]{\cvPath SPGpCDSS/const-0.9-1000/Comp_Sim-NSPG+PGCCD+CPSI1_Comparison_const-0.9-1000.png}}\\
  \subbottom[NSPG-CPSI1: Tempo.]{%
    \includegraphics[width=0.32\linewidth]{\cvPath SPG/const-0.9-1000/Comp_Time-NSPG+CPSI1_Comparison_const-0.9-1000.png}}
  \subbottom[L0Learn-CPSI1: Tempo.]{%
    \includegraphics[width=0.32\linewidth]{\cvPath L0LearnPSI1/const-0.9-1000/Comp_Time-L0LearnCPSI1_Comparison_const-0.9-1000.png}}
  \subbottom[NSPG-PGCCD-CPSI1: Tempo.]{%
    \includegraphics[width=0.32\linewidth]{\cvPath SPGpCDSS/const-0.9-1000/Comp_Time-NSPG+PGCCD+CPSI1_Comparison_const-0.9-1000.png}}
  \caption{Comparação das estratégias de validação cruzada (Correlação constante, $\rho=0.9,\ p=1000,\ k^\dagger=20,\ \text{SNR}=5$).}
  \label{fig:cv_comp_app2}
\end{figure}

\begin{figure}[ht]
  \centering
  \subbottom[NSPG-CPSI1: Similaridade.]{%
    \includegraphics[width=0.32\linewidth]{\cvPath SPG/exp-0.9-1000/Comp_Sim-NSPG+CPSI1_Comparison_exp-0.9-1000.png}}
  \subbottom[L0Learn-CPSI1: Similaridade.]{%
    \includegraphics[width=0.32\linewidth]{\cvPath L0LearnPSI1/exp-0.9-1000/Comp_Sim-L0LearnCPSI1_Comparison_exp-0.9-1000.png}}
  \subbottom[NSPG-PGCCD-CPSI1: Similaridade.]{%
    \includegraphics[width=0.32\linewidth]{\cvPath SPGpCDSS/exp-0.9-1000/Comp_Sim-NSPG+PGCCD+CPSI1_Comparison_exp-0.9-1000.png}}\\
  \subbottom[NSPG-CPSI1: Tempo.]{%
    \includegraphics[width=0.32\linewidth]{\cvPath SPG/exp-0.9-1000/Comp_Time-NSPG+CPSI1_Comparison_exp-0.9-1000.png}}
  \subbottom[L0Learn-CPSI1: Tempo.]{%
    \includegraphics[width=0.32\linewidth]{\cvPath L0LearnPSI1/exp-0.9-1000/Comp_Time-L0LearnCPSI1_Comparison_exp-0.9-1000.png}}
  \subbottom[NSPG-PGCCD-CPSI1: Tempo.]{%
    \includegraphics[width=0.32\linewidth]{\cvPath SPGpCDSS/exp-0.9-1000/Comp_Time-NSPG+PGCCD+CPSI1_Comparison_exp-0.9-1000.png}}
  \caption{Comparação das estratégias de validação cruzada (Correlação exponencial, $\rho=0.9,\ p=1000,\ k^\dagger=20,\ \text{SNR}=5$).}
  \label{fig:cv_comp_app3}
\end{figure}

\chapter{Comparação adicional com L0Learn}
\label{chap:anexos_l0learn}

Neste apêndice, apresentamos comparações adicionais entre o PGCCD e o L0Learn para diferentes cenários.

\begin{figure}[ht]
  \centering
  \subbottom[Similaridade do Suporte.]{%
    \includegraphics[width=0.45\linewidth]{\mixedPath const-0.5-2000/pnSim-NewTerminal_mixed_v2_const_0.5_p2000.png}}
  \subbottom[Tempo de execução (s).]{%
    \includegraphics[width=0.45\linewidth]{\mixedPath const-0.5-2000/pnTime-NewTerminal_mixed_v2_const_0.5_p2000.png}}\\
  \subbottom[Erro Máximo nos Parâmetros ($L_\infty$).]{%
    \includegraphics[width=0.45\linewidth]{\mixedPath const-0.5-2000/pnInf-NewTerminal_mixed_v2_const_0.5_p2000.png}}
  \caption{Comparação entre os métodos propostos e variantes do L0Learn. Correlação constante, $\rho=0.5,\ p=2000,\ k^\dagger=100,\ \text{SNR}=10$.}
  \label{fig:l0learn_anexo_1}
\end{figure}

\begin{figure}[ht]
  \centering
  \subbottom[Similaridade do Suporte.]{%
    \includegraphics[width=0.45\linewidth]{\mixedPath const-0.9-1000/pnSim-NewTerminal_mixed_v2_const_0.9_p1000.png}}
  \subbottom[Tempo de execução (s).]{%
    \includegraphics[width=0.45\linewidth]{\mixedPath const-0.9-1000/pnTime-NewTerminal_mixed_v2_const_0.9_p1000.png}}\\
  \subbottom[Erro Máximo nos Parâmetros ($L_\infty$).]{%
    \includegraphics[width=0.45\linewidth]{\mixedPath const-0.9-1000/pnInf-NewTerminal_mixed_v2_const_0.9_p1000.png}}
  \caption{Comparação entre os métodos propostos e variantes do L0Learn. Correlação constante, $\rho=0.9,\ p=1000,\ k^\dagger=20,\ \text{SNR}=5$.}
  \label{fig:l0learn_anexo_2}
\end{figure}

\begin{figure}[ht]
  \centering
  \subbottom[Similaridade do Suporte.]{%
    \includegraphics[width=0.45\linewidth]{\mixedPath exp-0.9-1000/pnSim-NewTerminal_mixed_v2_exp_0.9_p1000.png}}
  \subbottom[Tempo de execução (s).]{%
    \includegraphics[width=0.45\linewidth]{\mixedPath exp-0.9-1000/pnTime-NewTerminal_mixed_v2_exp_0.9_p1000.png}}\\
  \subbottom[Erro Máximo nos Parâmetros ($L_\infty$).]{%
    \includegraphics[width=0.45\linewidth]{\mixedPath exp-0.9-1000/pnInf-NewTerminal_mixed_v2_exp_0.9_p1000.png}}
  \caption{Comparação entre os métodos propostos e variantes do L0Learn. Correlação exponencial, $\rho=0.9,\ p=1000,\ k^\dagger=20,\ \text{SNR}=5$.}
  \label{fig:l0learn_anexo_3}
\end{figure}
\chapter{Análise do Refinamento}
\label{chap:anexos_refinement}

Neste apêndice, detalhamos o impacto do refinamento final na qualidade da solução. Os gráficos abaixo mostram, para cada método e cenário, a frequência com que o candidato a partir do zero (\textit{zero-based}) foi escolhido em detrimento do candidato do caminho (\textit{path-based}) (barra ``Wins'', que contabiliza vitórias, empates e derrotas na similaridade quando o zero é escolhido), e o impacto dessa escolha na similaridade do suporte (``Sim Improv''). A barra agrupa visualmente ``Wins'' (melhora no suporte), ``Ties'' (igualdade) e ``Losses'' (piora no suporte) condicionados à escolha pelo zero.

\begin{figure}[ht]
  \centering
  \subbottom[NSPG+CPSI(1): Refinamento.]{%
    \includegraphics[width=0.49\linewidth]{\cvPath SPG/exp-0.5-2000/Comp_Refinement-NSPG+CPSI1_Comparison_exp-0.5-2000.png}}
  \subbottom[L0Learn+CPSI(1) val: Refinamento.]{%
    \includegraphics[width=0.49\linewidth]{\cvPath L0LearnPSI1/exp-0.5-2000/Comp_Refinement-L0LearnCPSI1_Comparison_exp-0.5-2000.png}}\\
  \subbottom[NSPG+PGCCD+CPSI(1): Refinamento.]{%
    \includegraphics[width=0.49\linewidth]{\cvPath SPGpCDSS/exp-0.5-2000/Comp_Refinement-NSPG+PGCCD+CPSI1_Comparison_exp-0.5-2000.png}}
  \caption{Impacto do refinamento final (Correlação exponencial, $\rho=0.5,\ p=2000,\ k^\dagger=100,\ \text{SNR}=10$).}
  \label{fig:refinement_exp_0.5}
\end{figure}

\begin{figure}[ht]
  \centering
  \subbottom[NSPG+CPSI(1): Refinamento.]{%
    \includegraphics[width=0.49\linewidth]{\cvPath SPG/const-0.5-2000/Comp_Refinement-NSPG+CPSI1_Comparison_const-0.5-2000.png}}
  \subbottom[L0Learn+CPSI(1) val: Refinamento.]{%
    \includegraphics[width=0.49\linewidth]{\cvPath L0LearnPSI1/const-0.5-2000/Comp_Refinement-L0LearnCPSI1_Comparison_const-0.5-2000.png}}\\
  \subbottom[NSPG+PGCCD+CPSI(1): Refinamento.]{%
    \includegraphics[width=0.49\linewidth]{\cvPath SPGpCDSS/const-0.5-2000/Comp_Refinement-NSPG+PGCCD+CPSI1_Comparison_const-0.5-2000.png}}
  \caption{Impacto do refinamento final (Correlação constante, $\rho=0.5,\ p=2000,\ k^\dagger=100,\ \text{SNR}=10$).}
  \label{fig:refinement_const_0.5}
\end{figure}

\chapter{Detalhes de Implementação do PGCCD}
\label{chap:anexos_implementacao}

O algoritmo PGCCD emprega a estratégia de \textit{Active Set} conforme descrito em \cite{fastselect}. O método executa ciclos completos de descida coordenada até que o suporte da solução se mantenha inalterado por 10 iterações consecutivas (\texttt{ActiveSetNum=10}). A partir desse ponto, as iterações são restritas apenas às variáveis não-nulas (conjunto ativo) até a convergência. Por fim, uma verificação de otimalidade é realizada em todas as variáveis para garantir que nenhuma coordenada fora do suporte viola as condições de mínimo local. Adicionalmente, para reduzir o custo computacional nas fases iniciais, utilizamos uma estratégia de ordenação parcial (\textit{partial greedy sort}), onde somente os 25\% das variáveis com maior correlação com o resíduo são ordenadas para a varredura gulosa.
